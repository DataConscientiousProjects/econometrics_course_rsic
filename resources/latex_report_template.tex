% Assignment report LaTeX template
% for class Geospatial Modeling and Analysis
% GIS/MEA582-601 Spring 2014


\documentclass[10pt]{article}
\usepackage[utf8]{inputenc}
% comment the following line when compiling the real document (it causes demo mode where images are black)
\usepackage[demo]{graphicx}
% uncomment the following line when compiling the real document
% \usepackage{graphicx}
\usepackage[colorlinks=false]{hyperref}
\usepackage{caption}
\usepackage{subcaption}
\usepackage{mdwlist}
\usepackage{booktabs}
\usepackage[american]{babel}
\usepackage{csquotes}
\usepackage{textcomp}

\usepackage[left=1in, right=1in]{geometry}


% % % % % % % % % % % % % % % % % % % % % % % % % % % % % % % % % % % % % % % %
\newcommand{\gmodule}[1]{\href{http://grass.osgeo.org/grass70/manuals/html70_user/#1.html}{\emph{#1}}}
\newcommand{\asixmodule}[1]{\emph{#1}}
\newcommand{\asevenmodule}[1]{\emph{#1}}
\newcommand{\module}[1]{\emph{#1}}
\newcommand{\grasslink}{\href{http://grass.osgeo.org/}{GRASS GIS}}


% % % % % % % % % % % % % % % % % % % % % % % % % % % % % % % % % % % % % % % %
\newcommand{\authorName}{Your Name}
\newcommand{\classSemestr}{Spring 2014}
\newcommand{\classId}{GIS/MEA582-601}
\newcommand{\className}{Geospatial Modeling and Analysis}
\newcommand{\classWeb}{\url{http://courses.ncsu.edu/mea582/common/GIS_anal_assign/GIS_Anal_Assignall.html}}
\newcommand{\assignmentTopic}{Assignment report \LaTeX{} template}
% % % % % % % % % % % % % % % % % % % % % % % % % % % % % % % % % % % % % % % %


% % fancyhdr
% % accoding to http://en.wikibooks.org/wiki/LaTeX/Page_Layout#Customising_with_fancyhdr
\usepackage{fancyhdr}
\setlength{\headheight}{15.2pt}
% 
% % default fancy page style
\fancyhf{}
\fancyhead[R]{\textit{\classId, \classSemestr, \authorName}}

\fancyfoot[R]{\thepage}
% 
% % plain affects chapter first page
\fancypagestyle{plain}{
\fancyhead{}
\renewcommand{\headrulewidth}{0pt}
\renewcommand{\footrulewidth}{0pt}
}
% 
\title{\classId, \classSemestr \\ \className \\ \assignmentTopic}
\author{\authorName}
\date{\today}
% 
% 
\hypersetup{
    pdftitle={\assignmentTopic},
    pdfsubject={\classId, \className, \classSemestr},
    pdfauthor={\authorName},
    pdfkeywords={GIS} {GRASS} {GRASS GIS} {open source}
                {free and open source} {ArcGIS} {ArcMap} {ESRI},
}




% rules for floats
% setting rules which are much less strict than default

% general parameters for all pages
\renewcommand{\topfraction}{0.9}    % max fraction of floats at top
\renewcommand{\bottomfraction}{0.8} % max fraction of floats at bottom

% parameters for text pages
\setcounter{topnumber}{2}
\setcounter{bottomnumber}{2}
\setcounter{totalnumber}{4}  % 2 may work better
\setcounter{dbltopnumber}{2}  % for 2-column pages
\renewcommand{\dbltopfraction}{0.9}  % fit big float above 2-col. text
\renewcommand{\textfraction}{0.07}  % allow minimal text w. figs

% parameters for float pages (pages without text)
\renewcommand{\floatpagefraction}{0.9} % require fuller float pages
% note that floatpagefraction must be less than topfraction
\renewcommand{\dblfloatpagefraction}{0.7}  % require fuller float pages
% remember to use [htp] or [htpb] for placement
% if you want all rules to apply


% image size commands
\newcommand{\oneimgwidth}{0.7\textwidth}
\newcommand{\twoimgwidth}{0.45\textwidth}
\newcommand{\threeimgwidth}{0.3\textwidth}
\newcommand{\twobytwoimgwidth}{0.4\textwidth}


\begin{document}


\maketitle
\noindent
\rule{\textwidth}{1.5pt}

% will not show header for the fist page but that's actually what we want now
\pagestyle{fancy}


% % % % % % % % % % % % % % % % % % % % % % % % % % % % % % % % % % % % % % % %
\section*{Introduction}

Task, motivation

% % % % % % % % % % % % % % % % % % % % % % % % % % % % % % % % % % % % % % % %
\section*{Approach}

We list the raster and vector data using the layer management tools and display
the selected layers at appropriate spatial extent and resolution.
We set the spatial extent for our study area and define resolution at 30m\ldots\ and
display raster and vector data using the default color maps and basic line symbols.
We use the metadata information stored with each map layer to identify
the original raster data resolution, spatial extent and range of values\ldots


% % % % % % % % % % % % % % % % % % % % % % % % % % % % % % % % % % % % % % % %
\section*{Results}

The data set includes raster and vector data layers for North Carolina
at several spatial extents and resolutions\ldots

The elevation and road map for the study area is in the Fig. 1\ldots\ The state wide DEM
and climate data points are in Fig. 10\ldots \emph{(See the page \pageref{fig:figure-internal-label} and further
for suggestion on how to arrange your figures. Two images in row are usually the right choice,
e.g. figure \ref{fig:twotwo})}


% % % % % % % % % % % % % % % % % % % % % % % % % % % % % % % % % % % % % % % %
\section*{Discussion}

Displaying the data in both systems was straightforward. The default color table
for the elevation data was fuzzy, classification into fewer discrete classes
in ArcGIS or conversion to integers brings out the contours (choropleths?).
More detailed representation was achieved by relief shading.
I could get GRASS to display points using the circle symbol --- I was getting the following error.
It was impossible to find where to set the resolution when displaying the elevation data in ArcGIS\ldots


% % % % % % % % % % % % % % % % % % % % % % % % % % % % % % % % % % % % % % % %
\section*{What I learned}

I got familiar with ArcGIS basics, and with the data set. I already GRASS GIS know pretty well\ldots



% % % % % % % % % % % % % % % % % % % % % % % % % % % % % % % % % % % % % % % %
% figures
\newcommand{\figuresPath}{../figures}
% ../figures counts with directory layout:
% - hw1
%    - figures
%    - report
% the other posibilities are ./figures (tex file in the same directory as figures directory)
% and . (tex file and all figures in the same directory)
% command should be used for all image paths to ensure consistecy
% however, hardcoding of the path is possible as well

% templates to include figures
% uncomment lines with caption and label to add/exclude caption (text) or label (figure reference)
\begin{figure}[htbp]
  \centering
  \includegraphics[width=\oneimgwidth]{\figuresPath/picture_name}
  \caption{Image description}
  \label{fig:figure-internal-label}
\end{figure}

\begin{figure}[htbp]
  \centering
  \begin{subfigure}[b]{\twoimgwidth}
    \includegraphics[width=\textwidth]{\figuresPath/picture_name}
    \caption{Left figure}
%     \label{fig:}
  \end{subfigure}%
  ~ % ~, \quad, \qquad etc. or \\
  \begin{subfigure}[b]{\twoimgwidth}
    \includegraphics[width=\textwidth]{../figures/picture_name}
    \caption{Right figure}
%     \label{fig:}
  \end{subfigure}
  \caption{Main figure description}
  \label{fig:twotwo}
\end{figure}

\begin{figure}[htbp]
  \centering
  \begin{subfigure}[b]{\twobytwoimgwidth}
    \includegraphics[width=\textwidth]{}
    \caption{}
%     \label{fig:}
  \end{subfigure}%
  ~ % ~, \quad, \qquad etc. or \\
  \begin{subfigure}[b]{\twobytwoimgwidth}
    \includegraphics[width=\textwidth]{}
    \caption{}
%     \label{fig:}
  \end{subfigure}
  \\[0.01\textheight]
    \begin{subfigure}[b]{\twobytwoimgwidth}
    \includegraphics[width=\textwidth]{}
    \caption{}
%     \label{fig:}
  \end{subfigure}
  ~
  \begin{subfigure}[b]{\twobytwoimgwidth}
    \includegraphics[width=\textwidth]{}
    \caption{}
%     \label{fig:}
  \end{subfigure}%
  \caption{}
%   \label{fig:}
\end{figure}


\begin{figure}[htbp]
  \centering
  \begin{subfigure}[b]{\threeimgwidth}
    \includegraphics[width=\textwidth]{}
    \caption{left}
%     \label{fig:}
  \end{subfigure}%
  ~ % ~, \quad, \qquad etc. or \\
    \begin{subfigure}[b]{\threeimgwidth}
    \includegraphics[width=\textwidth]{}
    \caption{middle}
%     \label{fig:}
  \end{subfigure}
  ~
  \begin{subfigure}[b]{\threeimgwidth}
    \includegraphics[width=\textwidth]{}
    \caption{right}
%     \label{fig:}
  \end{subfigure}%
%   \caption{}
%   \label{fig:}
\end{figure}


\end{document}
